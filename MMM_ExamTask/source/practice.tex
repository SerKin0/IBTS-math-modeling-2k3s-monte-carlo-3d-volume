\section{Практическая часть}

\subsection{Реализация метода Монте-Карла}

Метод Монте-Карло для нахождения объема тела заключается в следующем алгоритме:
\begin{enumerate}
	\item Генерируем \texttt{N} случайных значений для каждой оси пространства, в котором находится наша фигура, тем самым получая \texttt{N} точек;
	\item Считаем количество точек, которые попали в область фигуры:
	\item Находим по формуле объем нашего тела:
	\[V_\text{тела} = \frac{M}{N} V,\]
	где $V$ --- объем области, в которой считаем объем фигуры.
\end{enumerate}

Назовем функцию \texttt{monte\_carlo\_3rd}, то есть для трехмерного пространства. В качестве параметров функции будет вводить:
\begin{itemize}
	\item Функция, которая принимает координаты точки и возвращает \texttt{True}, если точка находится внутри фигуры, и \texttt{False} в противном случае;
	\item Пределы области по трем осям XYZ;
	\item Количество генерируемых точек;
\end{itemize}
\lstinputlisting[language=Python, caption={\texttt{monte\_carlo.py}}]{../monte_carlo.py}

\subsection{Объем тела}
Протестируем функцию на формуле \cref{eq:main} и на $1\ 000\ 000$ точек:

\lstinputlisting[language=Python, caption={\texttt{task\_16.py}}]{../task_16.py}

В результате получается значение:
\[V_{прак} = 11.248415999999999\]
\[\Delta V = |V_\text{прак} - V_\text{теор}| = 0.017355973094854704\]

\subsection{Оценка времени и точности вычислений}

Для проведения экспериментов на точность получаемых значений напишем программу, которая будет изменять количество случайно сгенерированных точек от 0 до $1\ 000\ 000$ с шагом $2\ 000$, чтобы уменьшить время вычисления, и вносить их в таблицу.

Чтобы было удобно обрабатывать данные, сразу будем вычислять:
\begin{itemize}
	\item Среднеквадратичное отклонение (\texttt{std});
	\item Среднее арифметическое значение (\texttt{mean});
	\item Минимальное значение (\texttt{min});
	\item Максимальное значение (\texttt{max});
	\item Случайную погрешность значений (\texttt{random\_error});
\end{itemize}
Объема и затраченного времени на вычисление его. Случайная погрешность вычисляется по формуле:
\begin{equation} \label{eq:11}
	\Delta V_\text{сл} = t_{\alpha, \infty} \cdot \frac{S_V}{3},
\end{equation}
где $\alpha$ --- доверительный коэффициент, возьмем его равным $\alpha = 0.95$, тогда $t_{0.95,\infty} = 1.96$.
\begin{figure}[H]
	\centering
	\includegraphics[width=\linewidth]{graph/graph_1}
	\caption{Зависимость среднего значения объема от количества сгенерированных случайных точек}
	\label{fig:2}
\end{figure}
\begin{figure}[H]
	\centering
	\includegraphics[width=\linewidth]{graph/graph_4}
	\caption{Зависимость отношения разницы среднего объема тела, полученного методом Монте-Карло, и теоретического от количества сгенерированных случайных точек}
	\label{fig:5}
\end{figure}
\begin{figure}[H]
	\centering
	\includegraphics[width=\linewidth]{graph/graph_2}
	\caption{Зависимость среднего значения времени от количества сгенерированных случайных точек}
	\label{fig:3}
\end{figure}
\begin{figure}[H]
	\centering
	\includegraphics[width=\linewidth]{graph/graph_3}
	\caption{Зависимость случайной погрешности от количества сгенерированных случайных точек}
	\label{fig:4}
\end{figure}

Из графиков (\cref{fig:2,fig:5}) видно, что увеличение числа генерируемых точек повышает точность вычислений. Это проявляется в уменьшении разницы между теоретическим значением и средним практическим результатом. Графики демонстрируют обратную зависимость: с ростом числа точек скорость улучшения точности снижается.

Время на вычисление объема (\cref{fig:3,fig:4}) увеличивается линейно. Это связано со сложностью алгоритма Монте-Карло, которая пропорциональна числу точек ($O(N)$).