\section{Вывод}

На графиках зависимости точности и времени вычислений от числа сгенерированных точек наблюдается ожидаемое поведение метода Монте-Карло:
\begin{itemize}
	\item С увеличением количества точек \textbf{погрешность расчёта объёма уменьшается}, приближаясь к теоретическому значению. При этом скорость снижения погрешности замедляется, что соответствует зависимости \(O(1/\sqrt{N})\).
	\item \textbf{Время вычислений растёт линейно} с увеличением числа точек, что отражает вычислительную сложность алгоритма \(O(N)\).
\end{itemize}
Таким образом, в ходе работы были успешно решены все поставленные задачи:
\begin{enumerate}
	\item Разработана программная реализация метода Монте-Карло для вычисления объёма трёхмерного тела.
	\item Вычислен объём заданной области с высокой точностью.
	\item Проведён анализ точности и времени работы метода, подтвердивший его эффективность и предсказуемость.
\end{enumerate}