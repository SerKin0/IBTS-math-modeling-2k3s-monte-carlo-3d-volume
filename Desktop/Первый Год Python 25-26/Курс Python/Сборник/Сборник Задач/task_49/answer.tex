\answersection
Для решения задачи необходимо вывести символы строки, находящиеся на позициях с чётными индексами (0, 2, 4, ...), каждый на отдельной строке.

Сначала считываем входные данные: строку \texttt{s}.

Далее используем цикл, чтобы пройти по всем символам строки. В Python есть несколько способов решить эту задачу:

\textbf{Способ 1: Цикл for с использованием range и шага 2}
Можно использовать функцию \texttt{range()} с начальным значением 0, конечным значением \texttt{len(s)} и шагом 2. Это позволит перебрать только чётные индексы:
\begin{verbatim}
	for i in range(0, len(s), 2):
	print(s[i])
\end{verbatim}

\textbf{Способ 2: Цикл for с проверкой индекса}
Можно перебрать все индексы от 0 до \texttt{len(s)-1} и выводить только символы с чётными индексами:
\begin{verbatim}
	for i in range(len(s)):
	if i % 2 == 0:
	print(s[i])
\end{verbatim}

\textbf{Способ 3: Использование среза строки}
В Python можно использовать срезы с шагом: \texttt{s[::2]} вернёт строку, состоящую из символов с чётными индексами. Затем можно вывести каждый символ этой строки на отдельной строке:
\begin{verbatim}
	result = s[::2]
	for char in result:
	print(char)
\end{verbatim}

Первый способ является наиболее эффективным и понятным для данной задачи.

\lstinputlisting[language=Python]{task_49/main.py}