\section{Теоретическая часть}

В качестве осциллятора используется электрический контур, состоящий из последовательно соединенных катушки индуктивности \(L\), конденсатора \(C\), резистора \(R\) и внешнего источника ЭДС (см. \cref{fig:1}). 
\begin{figure}[H]
	\centering
	\resizebox{0.4\textwidth}{!}{%
		\begin{tikzpicture}
			% Paths, nodes and wires:
			\draw (7, 4) to[european resistor, l={\normalsize $R$}] (9, 4);
			\draw (9, 2) to[american inductor, l={\normalsize $L$}] (7, 2);
			\draw (7, 4) to[capacitor, l_={\normalsize $C$}, label distance=0.08cm] (7, 2);
			\draw (9, 2) to[rmeter, l_={\normalsize $E(t)$}, label distance=0.08cm] (9, 4);
		\end{tikzpicture}
	}
	\caption{Схема}
	\label{fig:1}
\end{figure}

Целью работы является экспериментальное исследование колебательных процессов в электрическом контуре, состоящем из последовательно соединенных катушки индуктивности \( L \), конденсатора \( C \), резистора \( R \) и внешнего источника ЭДС (см. рис. 1). Такой контур, как и другие системы различной природы, в которых возможны колебания, иногда называют осцилляторами.

Выведем дифференциальное уравнение, описывающее процессы в последовательном колебательном контуре.

Запишем II закон Кирхгофа для нашего контура  
\begin{align} \label{eq:th:1}
	IR + U_C = E(t) + E_i,
\end{align}
где \( I \) — ток в контуре, \( R \) — общее сопротивление контура, \( U_C \) — разность потенциалов между пластинами конденсатора, \( E(t) \) — внешняя ЭДС, а  
\begin{align} \label{eq:th:2}
	E_i = -L \frac{dI}{dt}
\end{align}
-- ЭДС самоиндукции, возникающая на индуктивности.

Заряд \( q \) на обкладках конденсатора связан с напряжением \( U_C \) равенством:  
\begin{align} \label{eq:th:3}
	q = CU_C,
\end{align}

Так как за время \( dt \) заряд конденсатора изменяется на величину  
\( dq = I \cdot dt \), то можно записать:
\begin{align} \label{eq:th:4}
	I = \frac{dq}{dt} = C \frac{dU_c}{dt}
\end{align}

Подставив в уравнение \cref{eq:th:1} значениfя для \( E_i \) из \cref{eq:th:2}, для \( I \) из \cref{eq:th:4} и для \( U_c \) из \cref{eq:th:3}, получим искомое дифференциальное уравнение
\begin{align} \label{eq:th:5}
	L \frac{d^2 q}{dt^2} + R \frac{dq}{dt} + \frac{q}{C} = E(t), 
\end{align}
которое после деления на \( L \) и введения новых обозначений примет  
вид:
\begin{align} \label{eq:th:6}
	\ddot{q} + 2\delta \dot{q} + \omega_0^2 q = f_1(t)
\end{align}

Величина \( \delta = R/2L \) называется коэффициентом затухания,
\begin{align} \label{eq:th:7}
	\omega_0 = \sqrt{\frac{1}{LC}}
\end{align}
-- собственной частотой контура, а \( f_1(t) = E(t)/L \) -- внешней <<вынуждающей силой>>.

Продифференцировав (6) по \( t \), можно получить аналогичное  
уравнение и для тока в контуре:
\begin{align} \label{eq:th:6a}
	\ddot{I} + 2\delta \dot{I} + \omega_0^2 I = f_2(t), \tag{\ref{eq:th:6}a}
\end{align}

где \( f_2(t) = \dfrac{1}{L} \dfrac{dE}{dt} \) -- <<вынуждающая сила>> колебаний тока.

С математической точки зрения уравнение \cref{eq:th:6} (также как и \cref{eq:th:6a})  
является неоднородным линейным дифференциальным уравнением 2-го порядка с постоянными коэффициентами. Решение такого  
уравнения, как известно, можно представить в виде суммы общего решения соответствующего однородного уравнения:
\begin{align} \label{eq:th:8}
	\ddot{q} + 2\delta \dot{q} + \omega_0^2 q = 0
\end{align} 

и частного решения неоднородного. Уравнение (8) описывает пове-
дение осциллятора в отсутствие внешней ЭДС, т.е. так называемые  
собственные (свободные) колебания, а частное решение неоднород-