\section{Теоретическая часть}

В качестве осциллятора используется электрический контур, состоящий из последовательно соединенных катушки индуктивности \(L\), конденсатора \(C\), резистора \(R\) и внешнего источника ЭДС (\cref{fig:1}). 
\begin{figure}[H]
	\centering
	\resizebox{0.4\textwidth}{!}{%
		\begin{tikzpicture}
			% Paths, nodes and wires:
			\draw (7, 4) to[european resistor, l={$R$}] (9, 4);
			\draw (9, 2) to[american inductor, l={$L$}] (7, 2);
			\draw (7, 4) to[capacitor, l_={$C$}, label distance=0.08cm] (7, 2);
			\draw (9, 2) to[rmeter, l_={$\Epsilon (t)$}, label distance=0.08cm] (9, 4);
		\end{tikzpicture}
	}%
	\caption{Схема}
	\label{fig:1}
\end{figure}

